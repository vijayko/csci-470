%
% This template is borrowed from OCW MIT 6.006 Spring 2020 course.
%
\documentclass[12pt,twoside]{article}

\input{macros}
\newcommand{\theproblemsetnum}{1} % modify this number to homework number.

\title{CSCI 470 Homework 1}

\begin{document}

\handout{Homework \theproblemsetnum}

\setlength{\parindent}{0pt}
\medskip\hrulefill\medskip

Please write your solutions in the \LaTeX. You may use online compiler such as \href{https://www.overleaf.com}{Overleaf} or any other compiler you are comfortable with to write your solutions in the \LaTeX. 

\medskip\hrulefill

%%%%%%%%%%%%%%%%%%%%%%%%%%%%%%%%%%%%%%%%%%%%%%%%%%%%%
% See below for common and useful latex constructs. %
%%%%%%%%%%%%%%%%%%%%%%%%%%%%%%%%%%%%%%%%%%%%%%%%%%%%%

% Some useful commands:
% $f(x) = \Theta(x)$
% $T(x, y) \leq \log(x) + 2^y + \binom{2n}{n}$
% \ttt{code\_function}


% You can create unnumbered lists as follows:
% \begin{itemize}
%     \item First item in a list
%         \begin{itemize}
%             \item First item in a list
%                 \begin{itemize}
%                     \item First item in a list
%                     \item Second item in a list
%                 \end{itemize}
%             \item Second item in a list
%         \end{itemize}
%     \item Second item in a list
% \end{itemize}

% You can create numbered lists as follows:
% \begin{enumerate}
%     \item First item in a list
%     \item Second item in a list
%     \item Third item in a list
% \end{enumerate}

% You can write aligned equations as follows:
% \begin{align}
%     \begin{split}
%         (x+y)^3 &= (x+y)^2(x+y) \\
%                 &= (x^2+2xy+y^2)(x+y) \\
%                 &= (x^3+2x^2y+xy^2) + (x^2y+2xy^2+y^3) \\
%                 &= x^3+3x^2y+3xy^2+y^3
%     \end{split}
% \end{align}

% You can create grids/matrices as follows:
% \begin{align}
%     A =
%     \begin{bmatrix}
%         A_{11} & A_{21} \\
%         A_{21} & A_{22}
%     \end{bmatrix}
% \end{align}

\begin{problems}

\problem  % Problem 1

\problem % Problem 2
\begin{problemparts}
    \problempart
    \problempart
\end{problemparts}

% \newpage
\problem  % Problem 3

\begin{problemparts}
    \problempart 
    \problempart
    \problempart
    \problempart
\end{problemparts}

\problem  % Problem 4

\problem  % Problem 5

\problem  % Problem 6

\problem  % Problem 7

\problem  % Problem 8

\problem  % Problem 9

\problem  % Problem 10

\newpage
{\large
\textbf{Extra Credit}
}

\problem  % Problem 11

\begin{problemparts}
    \problempart 
    \problempart
    \problempart
\end{problemparts}

\problem  % Problem 12

\begin{problemparts}
    \problempart 
    \problempart
\end{problemparts}

\end{problems}

\end{document}
