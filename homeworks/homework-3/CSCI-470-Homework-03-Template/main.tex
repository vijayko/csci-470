%
% This template is borrowed from OCW MIT 6.006 Spring 2020 course.
%
\documentclass[12pt,twoside]{article}

%
% This template is borrowed from OCW MIT 6.006 Spring 2020 course.
%

\newcommand{\name}{}

\usepackage{amssymb}
\usepackage{amsmath}
\usepackage{graphicx}
\usepackage{latexsym}
\usepackage{times,url}
\usepackage{cprotect}
\usepackage{listings}
\usepackage{graphicx}
\usepackage[table]{xcolor}
\usepackage[letterpaper]{geometry}
\usepackage{tikz-qtree}
\usepackage{enumerate}
\usepackage{hyperref}
\usepackage{clrscode3e}
% \usepackage{algorithm}
% \usepackage{algpseudocode}

\newcommand{\profs}{Instructor: Vijay Chaudhary}
\newcommand{\subj}{CSCI 470}
\newcommand{\ttt}[1]{{\tt\small #1}}

\definecolor{dkgreen}{rgb}{0,0.6,0}
\definecolor{dkblue}{rgb}{.2,.2,1}
\definecolor{gray}{rgb}{0.5,0.5,0.5}
\definecolor{mauve}{rgb}{0.58,0,0.82}

\lstset{
  language=Python,
  aboveskip=1pc,
  belowskip=1pc,
  basicstyle={\footnotesize\ttfamily},
  numbers=left,
  showstringspaces=false,
  numberstyle={\tiny\color{gray}\ttfamily},
  keywordstyle={\color{dkblue}\ttfamily},
  commentstyle={\color{dkgreen}\ttfamily},
  stringstyle={\color{mauve}\ttfamily},
}

% \lstset{
%   language=Python,
%   aboveskip=1pc,
%   belowskip=1pc,
%   basicstyle={\bf\color{white}\ttfamily},
%   numbers=left,
%   showstringspaces=false,
%   numberstyle={\bf\small\color{lightgray}\ttfamily},
%   keywordstyle={\bf\color{cyan}\ttfamily},
%   commentstyle={\bf\color{green}\ttfamily},
%   stringstyle={\bf\color{mauve}\ttfamily},
% }

\tikzset{
  % every node/.style={minimum width=2em,draw,circle},
  % level 1/.style={sibling distance=2cm},
  level distance=1cm,
  edge from parent/.style=
  {draw,edge from parent path={(\tikzparentnode) -- (\tikzchildnode)}},
}
\usetikzlibrary{shapes}

\newif\ifHideSolutions
\newcommand{\solution}[1]{\color{dkgreen}\textbf{Solution: }#1\color{black}}
\newcommand{\commonmistakes}[1]{{\color{dkblue}\textbf{Common Mistakes: }#1}}
\newcommand{\rubric}[1]{\color{dkgreen}{\bf Rubric:} #1\color{black}}

% \HideSolutionsfalse
% \ifHideSolutions
%   \renewcommand{\solution}[1]{}
%   \renewcommand{\rubric}[1]{}
% \fi

\newlength{\toppush}
\setlength{\toppush}{2\headheight}
\addtolength{\toppush}{\headsep}

\newcommand{\htitle}[2]{\noindent\vspace*{-\toppush}\newline\parbox{\textwidth}
{\textit{CSCI 470 Fundamentals of Algorithms}\hfill\name\newline
Howard University \hfill #2\newline
\profs\hfill #1 \\[-3.5ex]\newline
\mbox{}\hrulefill\mbox{}}\vspace*{1ex}\mbox{}\newline
\begin{center}{\Large\bf #1}\end{center}}

\newcommand{\handout}[2]{\thispagestyle{empty}
 \markboth{#1}{#1}
 \pagestyle{myheadings}\htitle{#1}{#2}}

\newcommand{\lecture}[3]{\thispagestyle{empty}
 \markboth{Lecture #1: #2}{Lecture #1: #2}
 \pagestyle{myheadings}\htitle{Lecture #1: #2}{#3}}

\newcommand{\htitlewithouttitle}[2]{\noindent\vspace*{-\toppush}\newline\parbox{6.5in}
{\textit{Introduction to Algorithms}\hfill#2\newline
Massachusetts Institute of Technology \hfill 6.006\newline
\profs\hfill Handout #1\vspace*{-.5ex}\newline
\mbox{}\hrulefill\mbox{}}\vspace*{1ex}\mbox{}\newline}

\newcommand{\handoutwithouttitle}[2]{\thispagestyle{empty}
 \markboth{Handout \protect\ref{#1}}{Handout \protect\ref{#1}}
 \pagestyle{myheadings}\htitlewithouttitle{\protect\ref{#1}}{#2}}

\newcommand{\exam}[2]{% parameters: exam name, date
 \thispagestyle{empty}
 \markboth{\hspace{1cm}\subj\ #1\hspace{1in}Name\hrulefill\ \ }%
          {\subj\ #1\hspace{1in}Name\hrulefill\ \ }
 \pagestyle{myheadings}\examtitle{#1}{#2}
 \renewcommand{\theproblem}{Problem \arabic{problemnum}}
}
\newcommand{\examsolutions}[3]{% parameters: handout, exam name, date
 \thispagestyle{empty}
 \markboth{Handout \protect\ref{#1}: #2}{Handout \protect\ref{#1}: #2}
% \pagestyle{myheadings}\htitle{\protect\ref{#1}}{#2}{#3}
 \pagestyle{myheadings}\examsolutionstitle{\protect\ref{#1}} {#2}{#3}
 \renewcommand{\theproblem}{Problem \arabic{problemnum}}
}
\newcommand{\examsolutionstitle}[3]{\noindent\vspace*{-\toppush}\newline\parbox{6.5in}
{\textit{Fundamentals of Algorithms}\hfill#3\newline
Howard University \hfill 6.006\newline
%Singapore-MIT Alliance \hfill SMA5503\newline
\profs\hfill Handout #1\vspace*{-.5ex}\newline
\mbox{}\hrulefill\mbox{}}\vspace*{1ex}\mbox{}\newline
\begin{center}{\Large\bf #2}\end{center}}

\newcommand{\takehomeexam}[2]{% parameters: exam name, date
 \thispagestyle{empty}
 \markboth{\subj\ #1\hfill}{\subj\ #1\hfill}
 \pagestyle{myheadings}\examtitle{#1}{#2}
 \renewcommand{\theproblem}{Problem \arabic{problemnum}}
}

\makeatletter
\newcommand{\exambooklet}[2]{% parameters: exam name, date
 \thispagestyle{empty}
 \markboth{\subj\ #1}{\subj\ #1}
 \pagestyle{myheadings}\examtitle{#1}{#2}
 \renewcommand{\theproblem}{Problem \arabic{problemnum}}
 \renewcommand{\problem}{\newpage
 \item \let\@currentlabel=\theproblem
 \markboth{\subj\ #1, \theproblem}{\subj\ #1, \theproblem}}
}
\makeatother


\newcommand{\examtitle}[2]{\noindent\vspace*{-\toppush}\newline\parbox{6.5in}
{\textit{Introduction to Algorithms}\hfill#2\newline
Massachusetts Institute of Technology \hfill 6.006 Spring 2020\newline
%Singapore-MIT Alliance \hfill SMA5503\newline
\profs\hfill #1\vspace*{-.5ex}\newline
\mbox{}\hrulefill\mbox{}}\vspace*{1ex}\mbox{}\newline
\begin{center}{\Large\bf #1}\end{center}}

\newcommand{\grader}[1]{\hspace{1cm}\textsf{\textbf{#1}}\hspace{1cm}}

\newcommand{\points}[1]{[#1 points]\ }
\newcommand{\parts}[1]
{
  \ifnum#1=1
  (1 part)
  \else
  (#1 parts)
  \fi
  \ 
}

\newcommand{\bparts}{\begin{problemparts}}
\newcommand{\eparts}{\end{problemparts}}
\newcommand{\ppart}{\problempart}

%\newcommand{\lg} {lg\ }

\setlength{\oddsidemargin}{0pt}
\setlength{\evensidemargin}{0pt}
\setlength{\textwidth}{6.5in}
\setlength{\topmargin}{0in}
\setlength{\textheight}{8.5in}


\newcommand{\Spawn}{{\bf spawn} }
\newcommand{\Sync}{{\bf sync}}

\renewcommand{\cases}[1]{\left\{ \begin{array}{ll}#1\end{array}\right.}
\newcommand{\cif}[1]{\mbox{if $#1$}}
\newcommand{\cwhen}[1]{\mbox{when $#1$}}

\newcounter{problemnum}
\newcommand{\theproblem}{Problem \theproblemsetnum-\arabic{problemnum}}
\newenvironment{problems}{
        \begin{list}{{\bf \theproblem. \hspace*{0.5em}}}
        {\setlength{\leftmargin}{0em}
         \setlength{\rightmargin}{0em}
         \setlength{\labelwidth}{0em}
         \setlength{\labelsep}{0em}
         \usecounter{problemnum}}}{\end{list}}
\makeatletter
\newcommand{\problem}[1][{}]{\item \let\@currentlabel=\theproblem \textbf{#1}}
\makeatother

\newcounter{problempartnum}[problemnum]
\newenvironment{problemparts}{
        \begin{list}{{\bf (\alph{problempartnum})}}
        {\setlength{\leftmargin}{2.5em}
         \setlength{\rightmargin}{2.5em}
         \setlength{\labelsep}{0.5em}}}{\end{list}}
\newcommand{\problempart}{\addtocounter{problempartnum}{1}\item}

\newenvironment{truefalseproblemparts}{
        \begin{list}{{\bf (\alph{problempartnum})\ \ \ T\ \ F\hfil}}
        {\setlength{\leftmargin}{4.5em}
         \setlength{\rightmargin}{2.5em}
         \setlength{\labelsep}{0.5em}
         \setlength{\labelwidth}{4.5em}}}{\end{list}}

\newcounter{exercisenum}
\newcommand{\theexercise}{Exercise \theproblemsetnum-\arabic{exercisenum}}
\newenvironment{exercises}{
        \begin{list}{{\bf \theexercise. \hspace*{0.5em}}}
        {\setlength{\leftmargin}{0em}
         \setlength{\rightmargin}{0em}
         \setlength{\labelwidth}{0em}
         \setlength{\labelsep}{0em}
        \usecounter{exercisenum}}}{\end{list}}
\makeatletter
\newcommand{\exercise}{\item \let\@currentlabel=\theexercise}
\makeatother

\newcounter{exercisepartnum}[exercisenum]
%\newcommand{\problem}[1]{\medskip\mbox{}\newline\noindent{\bf Problem #1.}\hspace*{1em}}
%\newcommand{\exercise}[1]{\medskip\mbox{}\newline\noindent{\bf Exercise #1.}\hspace*{1em}}

\newenvironment{exerciseparts}{
        \begin{list}{{\bf (\alph{exercisepartnum})}}
        {\setlength{\leftmargin}{2.5em}
         \setlength{\rightmargin}{2.5em}
         \setlength{\labelsep}{0.5em}}}{\end{list}}
\newcommand{\exercisepart}{\addtocounter{exercisepartnum}{1}\item}


% Macros to make captions print with small type and 'Figure xx' in bold.
\makeatletter
\def\fnum@figure{{\bf Figure \thefigure}}
\def\fnum@table{{\bf Table \thetable}}
\let\@mycaption\caption
%\long\def\@mycaption#1[#2]#3{\addcontentsline{\csname
%  ext@#1\endcsname}{#1}{\protect\numberline{\csname 
%  the#1\endcsname}{\ignorespaces #2}}\par
%  \begingroup
%    \@parboxrestore
%    \small
%    \@makecaption{\csname fnum@#1\endcsname}{\ignorespaces #3}\par
%  \endgroup}
%\def\mycaption{\refstepcounter\@captype \@dblarg{\@mycaption\@captype}}
%\makeatother
\let\mycaption\caption
%\newcommand{\figcaption}[1]{\mycaption[]{#1}}

\newcounter{totalcaptions}
\newcounter{totalart}

\newcommand{\figcaption}[1]{\addtocounter{totalcaptions}{1}\caption[]{#1}}

% \psfigures determines what to do for figures:
%       0 means just leave vertical space
%       1 means put a vertical rule and the figure name
%       2 means insert the PostScript version of the figure
%       3 means put the figure name flush left or right
\newcommand{\psfigures}{0}
\newcommand{\spacefigures}{\renewcommand{\psfigures}{0}}
\newcommand{\rulefigures}{\renewcommand{\psfigures}{1}}
\newcommand{\macfigures}{\renewcommand{\psfigures}{2}}
\newcommand{\namefigures}{\renewcommand{\psfigures}{3}}

\newcommand{\figpart}[1]{{\bf (#1)}\nolinebreak[2]\relax}
\newcommand{\figparts}[2]{{\bf (#1)--(#2)}\nolinebreak[2]\relax}


\macfigures     % STATE

% When calling \figspace, make sure to leave a blank line afterward!!
% \widefigspace is for figures that are more than 28pc wide.
\newlength{\halffigspace} \newlength{\wholefigspace}
\newlength{\figruleheight} \newlength{\figgap}
\newcommand{\setfiglengths}{\ifnum\psfigures=1\setlength{\figruleheight}{\hruleheight}\setlength{\figgap}{1em}\else\setlength{\figruleheight}{0pt}\setlength{\figgap}{0em}\fi}
\newcommand{\figspace}[2]{\ifnum\psfigures=0\leavefigspace{#1}\else%
\setfiglengths%
\setlength{\wholefigspace}{#1}\setlength{\halffigspace}{.5\wholefigspace}%
\rule[-\halffigspace]{\figruleheight}{\wholefigspace}\hspace{\figgap}#2\fi}
\newlength{\widefigspacewidth}
% Make \widefigspace put the figure flush right on the text page.
\newcommand{\widefigspace}[2]{
\ifnum\psfigures=0\leavefigspace{#1}\else%
\setfiglengths%
\setlength{\widefigspacewidth}{28pc}%
\addtolength{\widefigspacewidth}{-\figruleheight}%
\setlength{\wholefigspace}{#1}\setlength{\halffigspace}{.5\wholefigspace}%
\makebox[\widefigspacewidth][r]{#2\hspace{\figgap}}\rule[-\halffigspace]{\figruleheight}{\wholefigspace}\fi}
\newcommand{\leavefigspace}[1]{\setlength{\wholefigspace}{#1}\setlength{\halffigspace}{.5\wholefigspace}\rule[-\halffigspace]{0em}{\wholefigspace}}

% Commands for including figures with macpsfig.
% To use these commands, documentstyle ``macpsfig'' must be specified.
\newlength{\macfigfill}
\makeatother
\newlength{\bbx}
\newlength{\bby}
\newcommand{\macfigure}[5]{\addtocounter{totalart}{1}
\ifnum\psfigures=2%
\setlength{\bbx}{#2}\addtolength{\bbx}{#4}%
\setlength{\bby}{#3}\addtolength{\bby}{#5}%
\begin{flushleft}
\ifdim#4>28pc\setlength{\macfigfill}{#4}\addtolength{\macfigfill}{-28pc}\hspace*{-\macfigfill}\fi%
\mbox{\psfig{figure=./#1.ps,%
bbllx=#2,bblly=#3,bburx=\bbx,bbury=\bby}}
\end{flushleft}%
\else\ifdim#4>28pc\widefigspace{#5}{#1}\else\figspace{#5}{#1}\fi\fi}
\makeatletter

\newlength{\savearraycolsep}
\newcommand{\narrowarray}[1]{\setlength{\savearraycolsep}{\arraycolsep}\setlength{\arraycolsep}{#1\arraycolsep}}
\newcommand{\normalarray}{\setlength{\arraycolsep}{\savearraycolsep}}

\newcommand{\hint}{{\bf Hint:\ }}

% Macros from /th/u/clr/mac.tex

\newcommand{\set}[1]{\left\{ #1 \right\}}
\newcommand{\abs}[1]{\left| #1\right|}
\newcommand{\card}[1]{\left| #1\right|}
\newcommand{\floor}[1]{\left\lfloor #1 \right\rfloor}
\newcommand{\ceil}[1]{\left\lceil #1 \right\rceil}
\newcommand{\ang}[1]{\ifmmode{\left\langle #1 \right\rangle}
   \else{$\left\langle${#1}$\right\rangle$}\fi}
        % the \if allows use outside mathmode,
        % but will swallow following space there!
\newcommand{\paren}[1]{\left( #1 \right)}
\newcommand{\bracket}[1]{\left[ #1 \right]}
\newcommand{\prob}[1]{\Pr\left\{ #1 \right\}}
\newcommand{\Var}{\mathop{\rm Var}\nolimits}
\newcommand{\expect}[1]{{\rm E}\left[ #1 \right]}
\newcommand{\expectsq}[1]{{\rm E}^2\left[ #1 \right]}
\newcommand{\variance}[1]{{\rm Var}\left[ #1 \right]}
\renewcommand{\choose}[2]{{{#1}\atopwithdelims(){#2}}}
\def\pmod#1{\allowbreak\mkern12mu({\rm mod}\,\,#1)}
\newcommand{\matx}[2]{\left(\begin{array}{*{#1}{c}}#2\end{array}\right)}
\newcommand{\Adj}{\mathop{\rm Adj}\nolimits}

\newtheorem{theorem}{Theorem}
\newtheorem{lemma}[theorem]{Lemma}
\newtheorem{corollary}[theorem]{Corollary}
\newtheorem{xample}{Example}
\newtheorem{definition}{Definition}
\newenvironment{example}{\begin{xample}\rm}{\end{xample}}
\newcommand{\proof}{\noindent{\em Proof.}\hspace{1em}}
\def\squarebox#1{\hbox to #1{\hfill\vbox to #1{\vfill}}}
\newcommand{\qedbox}{\vbox{\hrule\hbox{\vrule\squarebox{.667em}\vrule}\hrule}}
\newcommand{\qed}{\nopagebreak\mbox{}\hfill\qedbox\smallskip}
\newcommand{\eqnref}[1]{(\protect\ref{#1})}

%%\newcommand{\twodots}{\mathinner{\ldotp\ldotp}}
\newcommand{\transpose}{^{\mbox{\scriptsize \sf T}}}
\newcommand{\amortized}[1]{\widehat{#1}}

\newcommand{\punt}[1]{}

%%% command for putting definitions into boldface
% New style for defined terms, as of 2/23/88, redefined by THC.
\newcommand{\defn}[1]{{\boldmath\textit{\textbf{#1}}}}
\newcommand{\defi}[1]{{\textit{\textbf{#1\/}}}}

\newcommand{\red}{\leq_{\rm P}}
\newcommand{\lang}[1]{%
\ifmmode\mathord{\mathcode`-="702D\rm#1\mathcode`\-="2200}\else{\rm#1}\fi}

%\newcommand{\ckt}[1]{\ifmmode\mathord{\mathcode`-="702D\sc #1\mathcode`\-="2200}\else$\mathord{\mathcode`-="702D\sc #1\mathcode`\-="2200}$\fi}
\newcommand{\ckt}[1]{\ifmmode \sc #1\else$\sc #1$\fi}

%% Margin notes - use \notesfalse to turn off notes.
\setlength{\marginparwidth}{0.6in}
\reversemarginpar
\newif\ifnotes
\notestrue
\newcommand{\longnote}[1]{
  \ifnotes
    {\medskip\noindent Note: \marginpar[\hfill$\Longrightarrow$]
      {$\Longleftarrow$}{#1}\medskip}
  \fi}
\newcommand{\note}[1]{
  \ifnotes
    {\marginpar{\tiny \raggedright{#1}}}
  \fi}


\newcommand{\reals}{\mathbbm{R}}
\newcommand{\integers}{\mathbbm{Z}}
\newcommand{\naturals}{\mathbbm{N}}
\newcommand{\rationals}{\mathbbm{Q}}
\newcommand{\complex}{\mathbbm{C}}

\newcommand{\oldreals}{{\bf R}}
\newcommand{\oldintegers}{{\bf Z}}
\newcommand{\oldnaturals}{{\bf N}}
\newcommand{\oldrationals}{{\bf Q}}
\newcommand{\oldcomplex}{{\bf C}}

\newcommand{\w}{\omega}                 %% for fft chapter

\newenvironment{closeitemize}{\begin{list}
{$\bullet$}
{\setlength{\itemsep}{-0.2\baselineskip}
\setlength{\topsep}{0.2\baselineskip}
\setlength{\parskip}{0pt}}}
{\end{list}}

% These are necessary within a {problems} environment in order to restore
% the default separation between bullets and items.
\newenvironment{normalitemize}{\setlength{\labelsep}{0.5em}\begin{itemize}}
                              {\end{itemize}}
\newenvironment{normalenumerate}{\setlength{\labelsep}{0.5em}\begin{enumerate}}
                                {\end{enumerate}}

%\def\eqref#1{Equation~(\ref{eq:#1})}
%\newcommand{\eqref}[1]{Equation (\ref{eq:#1})}
\newcommand{\eqreftwo}[2]{Equations (\ref{eq:#1}) and~(\ref{eq:#2})}
\newcommand{\ineqref}[1]{Inequality~(\ref{ineq:#1})}
\newcommand{\ineqreftwo}[2]{Inequalities (\ref{ineq:#1}) and~(\ref{ineq:#2})}

\newcommand{\figref}[1]{Figure~\ref{fig:#1}}
\newcommand{\figreftwo}[2]{Figures \ref{fig:#1} and~\ref{fig:#2}}

\newcommand{\liref}[1]{line~\ref{li:#1}}
\newcommand{\Liref}[1]{Line~\ref{li:#1}}
\newcommand{\lirefs}[2]{lines \ref{li:#1}--\ref{li:#2}}
\newcommand{\Lirefs}[2]{Lines \ref{li:#1}--\ref{li:#2}}
\newcommand{\lireftwo}[2]{lines \ref{li:#1} and~\ref{li:#2}}
\newcommand{\lirefthree}[3]{lines \ref{li:#1}, \ref{li:#2}, and~\ref{li:#3}}

\newcommand{\lemlabel}[1]{\label{lem:#1}}
\newcommand{\lemref}[1]{Lemma~\ref{lem:#1}} 

\newcommand{\exref}[1]{Exercise~\ref{ex:#1}}

\newcommand{\handref}[1]{Handout~\ref{#1}}

\newcommand{\defref}[1]{Definition~\ref{def:#1}}

% (1997.8.16: Victor Luchangco)
% Modified \hlabel to only get date and to use handouts counter for number.
%   New \handout and \handoutwithouttitle commands in newmac.tex use this.
%   The date is referenced by <label>-date.
%   (Retained old definition as \hlabelold.)
%   Defined \hforcelabel to use an argument instead of the handouts counter.

\newcounter{handouts}
\setcounter{handouts}{0}

\newcommand{\hlabel}[2]{%
\stepcounter{handouts}
{\edef\next{\write\@auxout{\string\newlabel{#1}{{\arabic{handouts}}{0}}}}\next}
\write\@auxout{\string\newlabel{#1-date}{{#2}{0}}}
}

\newcommand{\hforcelabel}[3]{%          Does not step handouts counter.
\write\@auxout{\string\newlabel{#1}{{#2}{0}}}
\write\@auxout{\string\newlabel{#1-date}{{#3}{0}}}}


% less ugly underscore
% --juang, 2008 oct 05
\renewcommand{\_}{\vrule height 0 pt depth 0.4 pt width 0.5 em \,}

% multiline framed box (will always extend to the far right edge; for a short single line, use \fbox directly)
% --zabel, fall 2018
\newcommand\framepar[1]{\fbox{\begin{minipage}{\linewidth}#1\end{minipage}}}


\newcommand{\theproblemsetnum}{3} % modify this number to homework number.

\title{CSCI 470 Homework 3}

\begin{document}

\handout{Homework \theproblemsetnum: \ Solutions}

\setlength{\parindent}{0pt}
\medskip\hrulefill\medskip

{\bf Name:} ** Chad Toomer**\\
{\bf Student ID: } **@02965969** 

\medskip\hrulefill

%%%%%%%%%%%%%%%%%%%%%%%%%%%%%%%%%%%%%%%%%%%%%%%%%%%%%
% See below for common and useful latex constructs. %
%%%%%%%%%%%%%%%%%%%%%%%%%%%%%%%%%%%%%%%%%%%%%%%%%%%%%

% Some useful commands:
% $f(x) = \Theta(x)$
% $T(x, y) \leq \log(x) + 2^y + \binom{2n}{n}$
% \ttt{code\_function}


% You can create unnumbered lists as follows:
% \begin{itemize}
%     \item First item in a list
%         \begin{itemize}
%             \item First item in a list
%                 \begin{itemize}
%                     \item First item in a list
%                     \item Second item in a list
%                 \end{itemize}
%             \item Second item in a list
%         \end{itemize}
%     \item Second item in a list
% \end{itemize}

% You can create numbered lists as follows:
% \begin{enumerate}
%     \item First item in a list
%     \item Second item in a list
%     \item Third item in a list
% \end{enumerate}

% You can write aligned equations as follows:
% \begin{align}
%     \begin{split}
%         (x+y)^3 &= (x+y)^2(x+y) \\
%                 &= (x^2+2xy+y^2)(x+y) \\
%                 &= (x^3+2x^2y+xy^2) + (x^2y+2xy^2+y^3) \\
%                 &= x^3+3x^2y+3xy^2+y^3
%     \end{split}
% \end{align}

% You can create grids/matrices as follows:
% \begin{align}
%     A =
%     \begin{bmatrix}
%         A_{11} & A_{21} \\
%         A_{21} & A_{22}
%     \end{bmatrix}
% \end{align}


{\large 
\textbf{Sorting in Linear Time}}
\begin{problems}

\problem  % Problem 3.1
\begin{align*}
\frac{n}{2} \cdot \lg n - \frac{n}{2} \le \lg(n!) \le n \cdot \lg n - \frac{n^2}{2} - \frac{1}{2}
These bounds are asymptotically tight for \(\lg(n!)\).
\end{align*}

\problem  % Problem 3.3
\textbf{ENQUEUE(Q, x):}
\begin{align*}
    \text{if } & (\text{rear} + 1) \, \%\, \text{size} == \text{front}: \\
    & \quad \text{print("Error: Queue Overflow")} \\
    \text{else:} \\
    & \quad Q[\text{rear}] = x \\
    & \quad \text{rear} = (\text{rear} + 1) \, \%\, \text{size}
\end{align*}

\textbf{DEQUEUE(Q):}
\begin{align*}
    \text{if } & \text{front} == \text{rear}: \\
    & \quad \text{print("Error: Queue Underflow")} \\
    \text{else:} \\
    & \quad x = Q[\text{front}] \\
    & \quad \text{front} = (\text{front} + 1) \, \%\, \text{size} \\
    & \quad \text{return } x
\end{align*}

{\large 
\textbf{Stack and Queues}}

\problem  % Problem 3.4
\begin{document}

\textbf{HEAD-ENQUEUE(Q, x):}
\begin{aligned}
    & \text{if } (\text{head} - 1 + \text{size}) \% \text{size} == \text{tail}: \\
    & \quad \text{print("Error: Deque Overflow")} \\
    & \text{else:} \\
    & \quad \text{head} = (\text{head} - 1 + \text{size}) \% \text{size} \\
    & \quad \text{D[head]} = x
\end{aligned}


\textbf{TAIL-ENQUEUE(Q, x):}

\begin{aligned}
    & \text{if } (\text{tail} + 1) \% \text{size} == \text{head}: \\
    & \quad \text{print("Error: Deque Overflow")} \\
    & \text{else:} \\
    & \quad \text{D[tail]} = x \\
    & \quad \text{tail} = (\text{tail} + 1) \% \text{size}
\end{aligned}


\textbf{HEAD-DEQUEUE(Q):}

\begin{aligned}
    & \text{if head == tail:} \\
    & \quad \text{print("Error: Deque Underflow")} \\
    & \text{else:} \\
    & \quad x = \text{D[head]} \\
    & \quad \text{head} = (\text{head} + 1) \% \text{size} \\
    & \quad \text{return } x
\end{aligned}


\textbf{TAIL-DEQUEUE(Q):}

\begin{aligned}
    & \text{if head == tail:} \\
    & \quad \text{print("Error: Deque Underflow")} \\
    & \text{else:} \\
    & \quad \text{tail} = (\text{tail} - 1 + \text{size}) \% \text{size} \\
    & \quad \text{return } \text{D[tail]}
\end{aligned}


\end{document}

\problem  % Problem 3.5

{\large 
\textbf{Linked Lists and Binary Trees}}

\problem
\begin{document}

\textbf{Answer:} The given nonrecursive procedure for reversing a singly linked list of (\ n \) elements using constant storage runs in ( \Theta(n) \)-time.

\end{document}

\problem
\begin{document}

\textbf{Recursive Procedure to Print Keys of a Binary Tree:}

\begin{aligned}
    & \text{Print-Keys}(node): \\
    & \quad \text{if } node \neq \text{NULL:} \\
    & \quad \quad \text{Print-Keys}(node.\text{left}) \\
    & \quad \quad \text{Print-Keys}(node.\text{right}) \\
    & \quad \quad \text{print}(node.\text{key})
\end{aligned}

\end{document}

{\large 
\textbf{Hash Tables}}

\problem problem % 3.7
\begin{document}

\textbf{Conclusion:} To find the maximum element of a dynamic set ( S ) represented by a direct-address table ( T ) of length ( m ), we can iterate through the table ( T ) from index ( m ) down to 1 until we find the first non-empty slot, which corresponds to the maximum element in ( S ). The worst-case performance of this procedure is ( O(m) ).

\end{document}

\problem problem % 3.8
\begin{document}

\textbf{Answer:} When inserting the keys 5, 28, 19, 15, 20, 33, 12, 17, 10 into a hash table with 9 slots and resolving collisions by chaining, the resulting hash table would have chains at various slots. The distribution of keys within these chains would depend on their respective hash values, calculated using the hash function (h(k) = k \mod 9\). The exact arrangement of keys in the chains would depend on the specific values of the keys and their computed hash values.

\end{document}

\problem problem % 3.9
\begin{document}

\textbf{Answer:} For the given hash table of size ( m = 1000 ) and hash function ( h(k) = lfloor m(kA mod 1) rfloor ) with ( A = frac{\sqrt{5} - 1}{2} ), the locations to which the keys 61, 62, 63, 64, and 65 are mapped can be computed using the formula. The specific computation involves substituting each key into the formula to obtain the corresponding hash values. The resulting hash values would determine the locations in the hash table.

\end{document}

{\large 
\textbf{Binary Search Trees}}

\problem problem % 3.10

\begin{document}

\textbf{Difference between Binary Search Tree Property and Min-Heap Property:}
The binary search tree property dictates that for every node (x) in the tree, the keys in the left subtree of (x) are less than the key of (x), and the keys in the right subtree of (x) are greater than the key of (x). This ensures that an in-order traversal of the tree results in sorted keys.

On the other hand, the min-heap property states that for every node (x) in the tree, the key of (x) is less than or equal to the keys of its children. This property ensures that the minimum key is always at the root.

\textbf{Min-Heap Property and Printing Keys in Sorted Order:}
The min-heap property alone cannot be used to print out the keys of an (n)-node tree in sorted order in (O(n)) time. While the min-heap property guarantees that the root contains the minimum key, the keys in the left and right subtrees are not necessarily sorted in any specific order.

To print the keys in sorted order using the min-heap property, you would need to perform an in-order traversal, which typically takes (O(n log n)) time for a binary heap. This is because you would need to visit each node, and for each node, you may need to traverse its entire height to reach its children.

In contrast, the binary search tree property allows for an (O(n))-time in-order traversal, making it more suitable for efficiently printing out keys in sorted order.

\end{document}

\problem problem % 3.11
\begin{document}

\textbf{Preorder Tree Walk (Recursive):}
[
\begin{aligned}
    & \text{Preorder-Walk}(node): \\
    & \quad \text{if } node \neq \text{NULL}: \\
    & \quad \quad \text{print}(node.\text{key}) \\
    & \quad \quad \text{Preorder-Walk}(node.\text{left}) \\
    & \quad \quad \text{Preorder-Walk}(node.\text{right})
\end{aligned}
]

\textbf{Postorder Tree Walk (Recursive):}
[
\begin{aligned}
    & \text{Postorder-Walk}(node): \\
    & \quad \text{if } node \neq \text{NULL}: \\
    & \quad \quad \text{Postorder-Walk}(node.\text{left}) \\
    & \quad \quad \text{Postorder-Walk}(node.\text{right}) \\
    & \quad \quad \text{print}(node.\text{key})
\end{aligned}
]

\end{document}


\problem problem % 3.12
\usepackage{amsmath}

\begin{document}

\textbf{Recursive TREE-MINIMUM:}
[
\begin{aligned}
    & \text{Tree-Minimum}(node): \\
    & \quad \text{if } node.\text{left} = \text{NULL}: \\
    & \quad \quad \text{return } node \\
    & \quad \text{else:} \\
    & \quad \quad \text{return } \text{Tree-Minimum}(node.\text{left})
\end{aligned}
]

\textbf{Recursive TREE-MAXIMUM:}
[
\begin{aligned}
    & \text{Tree-Maximum}(node): \\
    & \quad \text{if } node.\text{right} = \text{NULL}: \\
    & \quad \quad \text{return } node \\
    & \quad \text{else:} \\
    & \quad \quad \text{return } \text{Tree-Maximum}(node.\text{right})
\end{aligned}
]

\end{document}


\problem problem % 3.13
\usepackage{amsmath}

\begin{document}

\textbf{TREE-PREDECESSOR:}
[
\begin{aligned}
    & \text{Tree-Predecessor}(node): \\
    & \quad \text{if } node.\text{left} \neq \text{NULL}: \\
    & \quad \quad \text{return } \text{Tree-Maximum}(node.\text{left}) \\
    & \quad \text{else:} \\
    & \quad \quad \text{let } \text{y} = \text{node.parent} \\
    & \quad \quad \text{while } \text{y} \neq \text{NULL} \text{ and } \text{node} = \text{y.left}: \\
    & \quad \quad \quad \text{node} = \text{y} \\
    & \quad \quad \quad \text{y} = \text{y.parent} \\
    & \quad \quad \text{return } \text{y}
\end{aligned}
]

\end{document}

\problem problem % 3.14
\usepackage{amsmath}

\begin{document}

\textbf{Recursive TREE-INSERT:}
[
\begin{aligned}
    & \text{Tree-Insert}(root, key): \\
    & \quad \text{if } root = \text{NULL}: \\
    & \quad \quad \text{return } \text{Node}(key) \\
    & \quad \text{else if } key < \text{root.key}: \\
    & \quad \quad \text{root.left} = \text{Tree-Insert}(\text{root.left}, key) \\
    & \quad \quad \text{root.left.parent} = \text{root} \\
    & \quad \text{else:} \\
    & \quad \quad \text{root.right} = \text{Tree-Insert}(\text{root.right}, key) \\
    & \quad \quad \text{root.right.parent} = \text{root} \\
    & \quad \text{return } \text{root}
\end{aligned}
]

\end{document}
This LaTeX code provides a recursive version of the TREE-INSERT procedure for inserting a key into a binary search tree. The code recursively traverses the tree based on the key values to find the appropriate position for the new node. If the key is smaller than the current node's key, it goes to the left subtree; otherwise, it goes to the right subtree. The procedure returns the modified tree after insertion. You can compile this code to generate a PDF document.







\newpage
{\large
\textbf{Extra Credit}
}

\problem problem % 3.15
\begin{document}

\textbf{Nonrecursive In-Order Tree Walk:}
[
\begin{aligned}
    & \text{Nonrecursive-InOrder-Walk(root):} \\
    & \quad \text{stack} = \text{EmptyStack()} \\
    & \quad \text{current} = \text{root} \\
    & \quad \text{while } \text{current} \neq \text{NULL} \text{ or stack is not empty:} \\
    & \quad \quad \text{while } \text{current} \neq \text{NULL:} \\
    & \quad \quad \quad \text{Push}(\text{stack, current}) \\
    & \quad \quad \quad \text{current} = \text{current.left} \\
    & \quad \quad \text{current} = \text{Pop}(\text{stack}) \\
    & \quad \quad \text{print}(\text{current.key}) \\
    & \quad \quad \text{current} = \text{current.right} \\
\end{aligned}
]

\end{document}

\problem problem % 3.16
\begin{document}

\textbf{Proof:}
Let (U) be the universe of keys with (|U| > nm), and we are storing a set of (n) keys into a hash table of size (m). We are using the hash function (h(k)) to map keys to slots in the hash table.

Consider the set (S) of all possible hash values:
[ S = \{h(k) mid k in U\} ]

Since (|U| > nm), and there are only (m) slots in the hash table, by the Pigeonhole Principle, there must exist at least one hash value (h) that corresponds to a subset of (U) with size at least (n). That is, there exists a subset (V \subseteq U) with (|V| \geq n) such that for all keys (k_1, k_2 \in V), (h(k_1) = h(k_2)).

Now, consider the worst-case scenario where all the keys in (V) are inserted into the hash table, and they all hash to the same slot. In this case, the chain in that slot will have (|V| \geq n) keys.

When searching for a key in this worst-case scenario, we may need to traverse the entire chain in that slot. Therefore, the worst-case searching time for hashing with chaining is (\Theta(n)).



\end{document}


\problem problem % 3.17

\begin{document}

[
\begin{aligned}
    & \text{DELETE}(T, z): \\
    & \quad \text{if } z.\text{left} = \text{NULL}: \\
    & \quad \quad \text{TRANSPLANT}(T, z, z.\text{right}) \\
    & \quad \text{else if } z.\text{right} = \text{NULL}: \\
    & \quad \quad \text{TRANSPLANT}(T, z, z.\text{left}) \\
    & \quad \text{else:} \\
    & \quad \quad \text{y} = \text{TREE-MINIMUM}(z.\text{right}) \\
    & \quad \quad \text{if } y.\text{parent} \neq z: \\
    & \quad \quad \quad \text{TRANSPLANT}(T, y, y.\text{right}) \\
    & \quad \quad \quad \text{y.right} = z.\text{right} \\
    & \quad \quad \quad \text{y.right.parent} = y \\
    & \quad \quad \text{TRANSPLANT}(T, z, y) \\
    & \quad \quad \text{y.left} = z.\text{left} \\
    & \quad \quad \text{y.left.parent} = y \\
\end{aligned}
]

\textbf{TRANSPLANT Procedure:}

[
\begin{aligned}
    & \text{TRANSPLANT}(T, u, v): \\
    & \quad \text{if } u.\text{parent} = \text{NULL}: \\
    & \quad \quad T.\text{root} = v \\
    & \quad \text{else if } u = u.\text{parent}.\text{left}: \\
    & \quad \quad u.\text{parent}.\text{left} = v \\
    & \quad \text{else:} \\
    & \quad \quad u.\text{parent}.\text{right} = v \\
    & \quad \text{if } v \neq \text{NULL}: \\
    & \quad \quad v.\text{parent} = u.\text{parent} \\
\end{aligned}
]

\end{document}

\end{problems}

\end{document}
