%
% This template is borrowed from OCW MIT 6.006 Spring 2020 course.
%
\documentclass[12pt,twoside]{article}

\input{macros}
\newcommand{\theproblemsetnum}{0} % Change this number to the homework number. For instance, now it's 0 for Homework 0

\title{CSCI-470 Homework 0}

\begin{document}

\handout{Homework \theproblemsetnum}

\setlength{\parindent}{0pt}
\medskip\hrulefill\medskip

{\bf Name:} ** Your Name**\\
{\bf Student ID: } **your student ID** 

\medskip\hrulefill

%%%%%%%%%%%%%%%%%%%%%%%%%%%%%%%%%%%%%%%%%%%%%%%%%%%%%
% See below for common and useful latex constructs. %
%%%%%%%%%%%%%%%%%%%%%%%%%%%%%%%%%%%%%%%%%%%%%%%%%%%%%

% Some useful commands:
% $f(x) = \Theta(x)$
% $T(x, y) \leq \log(x) + 2^y + \binom{2n}{n}$
% \ttt{code\_function}


% You can create unnumbered lists as follows:
% \begin{itemize}
%     \item First item in a list
%         \begin{itemize}
%             \item First item in a list
%                 \begin{itemize}
%                     \item First item in a list
%                     \item Second item in a list
%                 \end{itemize}
%             \item Second item in a list
%         \end{itemize}
%     \item Second item in a list
% \end{itemize}

% You can create numbered lists as follows:
% \begin{enumerate}
%     \item First item in a list
%     \item Second item in a list
%     \item Third item in a list
% \end{enumerate}

% You can write aligned equations as follows:
% \begin{align}
%     \begin{split}
%         (x+y)^3 &= (x+y)^2(x+y) \\
%                 &= (x^2+2xy+y^2)(x+y) \\
%                 &= (x^3+2x^2y+xy^2) + (x^2y+2xy^2+y^3) \\
%                 &= x^3+3x^2y+3xy^2+y^3
%     \end{split}
% \end{align}

% You can create grids/matrices as follows:
% \begin{align}
%     A =
%     \begin{bmatrix}
%         A_{11} & A_{21} \\
%         A_{21} & A_{22}
%     \end{bmatrix}
% \end{align}

\begin{problems}

\problem  % Problem 1

\begin{problemparts}
\problempart % Problem 1a
\problempart % Problem 1b
\problempart % Problem 1c
\end{problemparts}

\problem  % Problem 2

\newpage 
\problem  % Problem 3

\begin{problemparts}
\problempart % Problem 3a
\problempart % Problem 3b
\problempart % Problem 3c
\problempart % Problem 3d
\end{problemparts}

\problem  % Problem 4
Submit your implementation (Python program) on Canvas. Please use the 

\begin{lstlisting}
def insertion_sort(A):
    '''
    Input:  A     | Python List of positive integers
    Output: A | Python List of positive integers in sorted order 
    '''
    ##################
    # YOUR CODE HERE #
    ##################
    return A
\end{lstlisting}

\end{problems}

\end{document}
