%
% This template is borrowed from OCW MIT 6.006 Spring 2020 course.
%
\documentclass[12pt,twoside]{exam}

\input{macros}
\newcommand{\theproblemsetnum}{01} % modify this number to homework number.

\title{CSCI 470 Pseudocode Template}

\begin{document}

% \handout{Homework Template \theproblemsetnum}
\handout{\LaTeX \ Pseudocode Template} 

\setlength{\parindent}{0pt}
\medskip\hrulefill\medskip

Here is a template to write a pseudocode in \LaTeX using package ``clrscode3e". This package has already been included in ``macros.tex" file (in line 19). 


\begin{codebox}
    \Procname{$\proc{Insertion-Sort}(A)$}
        \li \For $j \gets 2$ \To $\attrib{A}{length}$
            \Do
            \li $\id{key} \gets A[j]$
            \li \Comment Insert $A[j]$ into the sorted sequence
            $A[1 \twodots j-1]$.
            \li $i \gets j-1$
            \li \While $i > 0$ and $A[i] > \id{key}$
                \Do
                \li $A[i+1] \gets A[i]$
                \li $i \gets i-1$
                \End
            \li $A[i+1] \gets \id{key}$
            \End
\end{codebox}

\begin{itemize}
    \item ``codebox" is used to write a procedure within it. 
    \item ``Procname" tag is used to initialize a procedure
    \item ``proc" tag is used to write the name of the procedure 
    \item ``Do" tag is used to start an indentation 
    \item `li' tag is used write a statement 
    \item keywords such as if, while, for, return, we just use the same keyword starting with an uppercase. Notice how ``\While", ``\For" are written in the pseudocode above. 
\end{itemize}

\end{document}
